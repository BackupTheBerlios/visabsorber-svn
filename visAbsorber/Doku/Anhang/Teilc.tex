% Anhang A - Quellcode

\chapter[Tex-Editoren und Distributionen]{Tex-Editoren und Distributionen}

\section{TeXnicCenter}
Als frei verf�gbarer Editor f�r Latex-Projekte unter Windows ist TeXnicCenter in Verbindung mit der Distribution Miktex zu empfehlen. Hierbei sollte zuerst Miktex und anschlie�end TeXnicCenter installiert werden. Um die vorhandene Vorlage fehlerfrei kompilieren zu k�nnen, mu� zus�tzlich zu den in TeXnicCenter bereits vordefinierten ein weiteres Ausgabeprofil angelegt werden. Ausgabeprofile dienen dazu festzulegen, welches Tex (Tex, Latex, PDFTex) benutzt wird und legen somit fest, in welchem Format verwendete Grafiken vorliegen m�ssen sowie auch das Format der Ausgabedatei. F�r weiterf�hrende Informationen sollten die entsprechenden Tutorials bem�ht werden.

\subsection{Ausgabeprofil f�r diese Vorlage}
Das zu erstellende Profil soll pdftex nutzen, um hilfreiche Optionen wie z.B. Querverweise im Zieldokument (pdf) zu erhalten. Um aber eps-Grafiken einbinden zu k�nnen, wird �ber den in der Vorlage vorhandenen Schalter $\backslash$pdfoutput=0 die PDF-Ausgabe unterdr�ckt, das Ergebnis des Kompilierens ist eine *.dvi-Datei. Diese muss anschlie�end zuerst in eine *.ps-Datei umgewandelt werden (mittels dvips.exe), welche anschlie�end in eine *.pdf-Datei konvertiert wird (ps2pdf.exe).

Im TeXnicCenter ist unter dem Men�punkt \textit{Ausgabe} der Punkt \textit{Ausgabeprofile definieren} (auch ALT+F7) anzuw�hlen. Dort wird dann das vordefinierte Profil LATEX-->PDF kopiert und mit einem neuen, sinnvollen Namen versehen, z.B. Diplomarbeit. Die Registerkarten \textit{(La)Tex} und \textit{Viewer} k�nnen unver�ndert �bernommen werden. Unter \textit{Nachbearbeitung} werden nun die o.g. Postprozessoren angelegt:

\begin{enumerate}

\item{DVIPS: Anwendung ist dvips.exe mit dem entsprechenden lokalen Pfad. Als Argument wird \textit{-R0 -P pdf ''\%Bm.dvi''} eingetragen.}

\item{PS2PDF: Anwendung ist gswin32c.exe (ghostview) ebenfalls mit dem entsprechenden lokalen Pfad. Als Argument wird \textit{-sPAPERSIZE=a4 -dSAFER -dBATCH -dNOPAUSE -sDEVICE=pdfwrite -sOutputFile=''\%bm.pdf'' -c save pop -f ''\%bm.ps''} angegeben.}

\end{enumerate}

Nun mu� das soeben angelegte Profil noch aktiviert werden (Menu \textit{Auswahl/Aktives Ausgabeprofil w�hlen}). 

Zur Unterst�tzung liegt der Vorlage ein entsprechendes Profil (Datei TexnicCenterProfil.tco) bei, das importiert werden kann. Die lokalen Pfade zu den Anwendungen und Viewern m�ssen allerdings �berpr�ft bzw. per Hand erg�nzt werden.

\section{Weitere Editoren/Distributionen}
Es existiert eine Vielzahl weiterer Editoren zum Erstellen von Tex-Dokumenten. Beispiele hierf�r sind WinEdt, Winshell, Latex Editor oder auch Lyx (wysiwyg). Eine �hnlich gro�e Auswahl herrscht bei Tex-Distributionen (z.B. tetex, auctex, emtex...). Eine gute �bersicht hierzu ist unter \url{http://www.math.vanderbilt.edu/~schectex/wincd/list_tex.htm} zu finden.